\documentclass[french,12pt,a4paper,openany]{book}
\usepackage[utf8]{inputenc}
\usepackage[T1]{fontenc}
\usepackage[francais]{babel}
\usepackage{hyperref}
\author{Kévin \textsc{Kerboit}, Pierre \textsc{Le Bras}\\Glen \textsc{Ollivier}, Valentin \textsc{Pearce}}
\title{Application de Suivi Sportif \\ Manuel d'utilisation}
\date{}
\begin{document}
	\maketitle
	\tableofcontents
	\chapter{Gestion des sportifs}
		\paragraph{}{Pour acceder à la gestion des sportifs, cliquez sur le bouton `Sportifs' du menu principal.}
		\section{Ajout d'un sportif}
			\paragraph{}{Depuis le menu de gestion des sportifs, cliquez sur le bouton `Ajouter'. Complètez ensuite le formulaire affiché.}
			\paragraph{}{Le pseudonyme ne peut pas être changé ultérireurement. Deux sportifs ne peuvent avoir la même combinaison de nom, prénom et date de naissance.}.
		\section{Modification d'un sportif}
			\paragraph{}{Depuis le menu de gestion des sportifs, cliquez sur le bouton `Modifier'.}
			\paragraph{}{Choisissez le sportif que vous voulez modifier dans le menu déroulant et cliquez sur le bouton `Choisir'. Le formulaire affiché contient les informations existantes.}
			\paragraph{}{Toutes les valeurs sont modifiables sauf le pseudonyme du sportif. Les contraintes restent les mêmes qu'à l'ajout d'un nouveau sportif.}.
		\section{Désactivation d'un sportif}
			\paragraph{}{Depuis le menu de gestion des sportifs, cliquez sur le bouton `Désactiver'.}
			\paragraph{}{Choisissez le sportif que vous voulez désactiver dans le menu déroulant et cliquez sur le bouton `Désactiver'.}
			\paragraph{}{Le sportif est désormais désactivé et ne recevra plus les liens vers les questionnaires.}.
		\section{Réactivation d'un sportif}
			\paragraph{}{Depuis le menu de gestion des sportifs, cliquez sur le bouton `Réactiver'.}
			\paragraph{}{Choisissez le sportif que vous voulez réactiver dans le menu déroulant et cliquez sur le bouton `Réactiver'.}
			\paragraph{}{Le sportif est désormais réactivé et recevra de nouveau les liens vers les questionnaires.}.
	\chapter{Gestion des questionnaires}
		\paragraph{}{Pour acceder à la gestion des questionnaires, cliquez sur le bouton `Questionnaires' du menu principal.}
		\section{Ajout d'un questionnaire}
			\paragraph{}{Depuis le menu de gestion des questionnaires, cliquez sur le bouton `Ajouter'.}
			\subsection{Description du questionnaire}
				\paragraph{}{Complètez le formulaire affiché.}
				\paragraph{}{La date de début doit être un Lundi et la date de fin doit être un Dimanche.}
				\paragraph{}{La date de début ne doit pas être passée et la date de fin doit être après la date début.}
				\paragraph{}{Validez les informations entrées dans le formulaire en cliquant sur le bouton `Ajouter des questions'.}
			\subsection{Ajouter des questions}
				\paragraph{}{Complètez le champ `Intitulé de la question'.}
				\paragraph{}{Pour ajouter une autre question, cliquez sur `Ajouter une question'.}
				\paragraph{}{Pour terminer la création du questionnaire, cliquez sur `Enregistrer le questionnaire'.}
		\section{Modification d'un questionnaire}
			\paragraph{}{Vous ne pouvez modifier que la date de fin d'un questionnaire déjà commencé.}
			\paragraph{}{Depuis le menu de gestion des questionnaires, cliquez sur le bouton `Modifier'.}
			\paragraph{}{Choisissez le questionnaire que vous voulez modifier dans le menu déroulant et cliquez sur le bouton `Choisir'. Le formulaire affiché contient la description du questionnaire.}
			\subsection{Modifier les informations du questionnaire}
				\paragraph{}{Modifez les champs du formulaire.}
				\paragraph{}{Si vous souhaitiez uniquement modifer les informations du formulaire, cliquez sur le bouton `Valider les modification'.}
			\subsection{Modifier les questions du questionnaire}
				\paragraph{}{Si vous souhaitiez modifer les questions du formulaire et leur ordre, cliquez sur le bouton `Modifier les questions'.}
				\paragraph{}{Choisissez la question que vous souhaitez modifer dans le menu déroulant. Si vous souhaitez la monter dans l'ordre des questions, appuyez sur le bouton `Monter', si vous souhaitez la descendre dans l'orde des questions, cliquez sur le bouton`Descendre'. Si vous souhaitez en modifier l'intitulé ou la réponse par défaut, cliquez sur le bouton `Modifier'.}
				\paragraph{}{Le formulaire affiché contient les valeurs courantes de la question. Pour valider le changements, cliquez sur le bouton `Valider'}


\end{document}
