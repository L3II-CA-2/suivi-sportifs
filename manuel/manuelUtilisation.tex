\documentclass[french,12pt,a4paper,openany]{book}
\usepackage[utf8]{inputenc}
\usepackage[T1]{fontenc}
\usepackage[francais]{babel}
\usepackage{hyperref}
\author{Kévin \textsc{Kerboit}, Pierre \textsc{Le Bras}\\Glen \textsc{Ollivier}, Valentin \textsc{Pearce}}
\date{}
\begin{document}

\title{Application de Suivi Sportif \\ Manuel d'utilisation}
\maketitle
\tableofcontents
\chapter{Gestion des sportifs}
\paragraph{}{Pour acceder à la gestion des sportifs, cliquez sur le bouton `Sportifs' du menu principal.}
\section{Ajout d'un sportif}
\paragraph{}{Depuis le menu de gestion des sportifs, cliquez sur le bouton `Ajouter'. Complètez ensuite le formulaire affiché.}
\paragraph{}{Le pseudonyme ne peut pas être changé ultérireurement. Deux sportifs ne peuvent avoir la même combinaison de nom, prénom et date de naissance.}.
\section{Modification d'un sportif}
\paragraph{}{Depuis le menu de gestion des sportifs, cliquez sur le bouton `Modifier'.}
\paragraph{}{Choisissez le sportif que vous voulez modifier dans le menu déroulant et cliquez sur le bouton `Choisir'. Le formulaire affiché contient les informations existantes.}
\paragraph{}{Toutes les valeurs sont modifiables sauf le pseudonyme du sportif. Les contraintes restent les mêmes qu'à l'ajout d'un nouveau sportif.}.
\section{Désactivation d'un sportif}
\paragraph{}{Depuis le menu de gestion des sportifs, cliquez sur le bouton `Désactiver'.}
\paragraph{}{Choisissez le sportif que vous voulez désactiver dans le menu déroulant et cliquez sur le bouton `Choisir'.}
\paragraph{}{Le sportif est désormais désactivé et ne recevra plus les liens vers les questionnaires.}.
\section{Réactivation d'un sportif}
\paragraph{}{Depuis le menu de gestion des sportifs, cliquez sur le bouton `Réactiver'.}
\paragraph{}{Choisissez le sportif que vous voulez réactiver dans le menu déroulant et cliquez sur le bouton `Choisir'.}
\paragraph{}{Le sportif est désormais réactivé et recevra de nouveau les liens vers les questionnaires.}.

\chapter{Gestion des questionnaires}
\end{document}
